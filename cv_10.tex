%%%%%%%%%%%%%%%%%%%%%%%%%%%%%%%%%%%%%%%%%
% Friggeri Resume/CV
% XeLaTeX Template
% Version 1.0 (5/5/13)
%
% This template has been downloaded from:
% http://www.LaTeXTemplates.com
%
% Original author:
% Adrien Friggeri (adrien@friggeri.net)
% https://github.com/afriggeri/CV
%
% License:
% CC BY-NC-SA 3.0 (http://creativecommons.org/licenses/by-nc-sa/3.0/)
%
% Important notes:
% This template needs to be compiled with XeLaTeX and the bibliography, if used,
% needs to be compiled with biber rather than bibtex.
%
%xelatex
%biber
%xelatex
%
%%%%%%%%%%%%%%%%%%%%%%%%%%%%%%%%%%%%%%%%%

\documentclass[]{friggeri-cv} % Add 'print' as an option into the square bracket to remove colors from this template for printing

\addbibresource{bibliography.bib} % Specify the bibliography file to include publications

\let\tempone\itemize
\let\temptwo\enditemize
\renewenvironment{itemize}{\tempone\addtolength{\itemsep}{0.5em}}{\temptwo}

\begin{document}

\header{cory}{schillaci}{Machine Learning and Data Science} % Your name and current job title/field
\footer{linkedin.com/in/coryschillaci\hspace{.25cm}{\small\bullet}\hspace{.25cm} github.com/coryschillaci}

%----------------------------------------------------------------------------------------
%	SIDEBAR SECTION
%----------------------------------------------------------------------------------------

\begin{aside} % In the aside, each new line forces a line break
\section{contact}
\href{mailto:coryschillaci@gmail.com}{coryschillaci@gmail.com}
~
+1 (206) 930-9189
%~
%\url{linkedin.com/in/coryschillaci}
~
1512 Bonita Avenue
Berkeley, CA 94709
\section{programming}
Python, C/C++
~
Pytorch, OpenCV, TensorRT, Pandas, Docker
\section{selected skills}
deep computer vision 
HMMs
robotics integration
\end{aside}

%----------------------------------------------------------------------------------------
%	PROFESSIONAL EXPERIENCE SECTION
%----------------------------------------------------------------------------------------

\section{professional experience}
%
\begin{entrylist}
\entry{Jan. 2018\\- present}
{Machine Learning Engineer}
{Saildrone}
{\vspace{-\baselineskip}
\begin{itemize}
\item Developed AI perception systems that integrate with unmanned wind-powered ocean vehicles (Saildrones) to provide realtime awareness of marine traffic and wildlife 
\item Trained heavily pruned deep object detection and image classification models in Pytorch for faster inference while maintaining high precision and recall
\item Deployed models to Jetson Nano/AGX platforms with TensorRT for optimal inference time performance
\item Managed third party data annotation of ~300k images taken by Saildrones at sea
\item Worked on all parts of the camera system including C++ integration with NVIDIA libraries such as LibArgus and TensorRT, platform integration with primary Saildrone operating system via gRPC, scalable deployment processes using SaltStack, end-to-end integration testing using CircleCI and real hardware, etc.
\end{itemize}
}
\smallskip
\entry{Aug. 2015 - \\Jan. 2018}
{Data Scientist}
{Streetline}
{\vspace{-\baselineskip}
\begin{itemize}
\item R\&D of machine learning algorithms to predict parking availability in real time from multi-modal sensor data including custom magnetic sensors, parking meters, cameras, and cell phone accelerometer/GPS data
\item Developed a modified time-dependent hidden Markov model and custom expectation-maximization training algorithms.
\item Interfaced with platform team to design a custom geospatial backend for data ingest, inference and display of final results. Worked with PostGIS, Open Street Maps, QGIS and other geospatial tools.
%\item Supported custom data analysis and data presentation needs of business, sales and client services teams.
\end{itemize}
}
\end{entrylist}

%----------------------------------------------------------------------------------------
%	EDUCATION SECTION
%----------------------------------------------------------------------------------------
\vspace{-\baselineskip}
\section{education}

\begin{entrylist}
%------------------------------------------------
\entry
{Dec. 2015}
{PhD, {\normalfont  Physics (Haxton Group)}}
{University of California, Berkeley}
{Thesis Title: Effective Interactions for Few-Body Physics. 
\begin{itemize}
\item Analytical simplification of three-body interactions in atomic nuclei 
\item Numerically determined spectra of spin-orbit interactions in Bose-Einstein condensates using C++ and Mathematica
\end{itemize}}
%------------------------------------------------
\entry
{June 2009}
{Bachelor of Science, {\normalfont Physics (Reinhardt Group)}}
{University of Washington}
{Magna cum laude with honors. Minors in mathematics and chemistry. 
\begin{itemize}
\item Built software in C with OpenMP and MPI to simulate dynamics of Bose-Einstein condensates using large-scale parallel computing resources.
\end{itemize}}
%------------------------------------------------
\end{entrylist}

%----------------------------------------------------------------------------------------
%	PUBLICATIONS SECTION
%----------------------------------------------------------------------------------------

%\section{publications}
%\newrefcontext[sorting=chronological]%
%    \nocite{*}%
%    \printbibliography[heading=none]%
%\printbibsection{article} % Print all articles from the bibliography


%----------------------------------------------------------------------------------------
%	RELEVANT COURSEWORK SECTION
%----------------------------------------------------------------------------------------

%\section{coursework}
%
%\begin{entrylist}
%%------------------------------------------------
%\entry
%{Spring 2014}
%{CS 289A: Introduction to Machine Learning}
%{UC Berkeley}
%{Theory and practice of classification and regression. Python implementation of models including logistic regression, random forests, neural networks, and boosting methods for classification of MNIST handwritten digits.} 
%%------------------------------------------------
%\entry
%{Spring 2014}
%{Info 290T: Data Mining}
%{UC Berkeley}
%{Practical aspects of data workflow including cleaning, ETL pipelines, distributed computing, modeling, and visualization. Taught by an industry professional.}
%%------------------------------------------------
%\entry
%{Fall 2014}
%{CS 281A: Statistical Learning Theory}
%{UC Berkeley}
%{Theoretical foundations of machine learning including detection and estimation theory, batch and incremental optimization, graphical models, exponential families.}
%%------------------------------------------------
%\entry
%{Jan. 2015}
%{Distributed Analytics \& ML with Apache Spark}
%{Berkeley Institute for Data Science}
%{Apache Spark and MLlib for distributed analysis of large data sets including click-through rate predictions and log analysis.}
%%------------------------------------------------
%\entry
%{In Progress}
%{CS 294: Deep Reinforcement Learning}
%{UC Berkeley}
%{Reinforcement learning with deep neural networks. Topics include imitation learning, policy gradient methods, and Q-learning. Self study.}
%%------------------------------------------------
%\end{entrylist}

%----------------------------------------------------------------------------------------
%	PROJECTS SECTION
%----------------------------------------------------------------------------------------
%\section{projects}
%
%\begin{entrylist}
%%------------------------------------------------
%\entry
%{2014}
%{Enhancing Wind Forecasts with Machine Learning}
%{}
%{Worked with Sail Tactics, LLC to apply machine learning to improve their forecast product.\begin{itemize}
%%\item Forecast data stored using PostgreSQL and PostGIS on Amazon S3
%%\item Extensive data acquisition and cleaning from government sources. 
%\item Improved RMSE of wind speed predictions by 20\% using time series analysis.
%\end{itemize}
%}
%%------------------------------------------------
%\entry
%{2015}
%{Sentiment Analysis and Stress Management (LiveJournal Posts)}
%{}
%{Working with Prof. John Canny to analyze sentiment and identify stress predictors from blog text.
%\begin{itemize}
%\item Experience with basic NLP pipeline including parsing using flex, linear models and neural networks, semantic analysis with word2vec, etc for 300GB of data. 
%\end{itemize}
%}
%%------------------------------------------------
%\end{entrylist}
%----------------------------------------------------------------------------------------
%	Skills SECTION
%----------------------------------------------------------------------------------------

%\section{skills}
%%Physics grad school allowed me to develop a strong background in mathematics, experience presenting technical concepts to all audiences, and the ability to acquire new knowledge quickly
%\textbf{general:} strong math background, experience critically assessing data and methodologies, rapid acquisition of new tools and concepts, technical presentations \\ 
%\textbf{programming languages:} Python, C/C++, Mathematica, Matlab, Julia\\
%\textbf{programming toolbox:} scikit-learn, Apache Spark, MLlib, git, SQL
%

%----------------------------------------------------------------------------------------
\end{document}