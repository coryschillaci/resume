%%%%%%%%%%%%%%%%%%%%%%%%%%%%%%%%%%%%%%%%%
% Friggeri Resume/CV
% XeLaTeX Template
% Version 1.0 (5/5/13)
%
% This template has been downloaded from:
% http://www.LaTeXTemplates.com
%
% Original author:
% Adrien Friggeri (adrien@friggeri.net)
% https://github.com/afriggeri/CV
%
% License:
% CC BY-NC-SA 3.0 (http://creativecommons.org/licenses/by-nc-sa/3.0/)
%
% Important notes:
% This template needs to be compiled with XeLaTeX and the bibliography, if used,
% needs to be compiled with biber rather than bibtex.
%
%xelatex
%biber
%xelatex
%
%%%%%%%%%%%%%%%%%%%%%%%%%%%%%%%%%%%%%%%%%

\documentclass[]{friggeri-cv} % Add 'print' as an option into the square bracket to remove colors from this template for printing

\addbibresource{bibliography.bib} % Specify the bibliography file to include publications

\begin{document}

\header{cory}{schillaci}{machine learning, analytics, and big data} % Your name and current job title/field
\footer{linkedin.com/in/coryschillaci}

%----------------------------------------------------------------------------------------
%	SIDEBAR SECTION
%----------------------------------------------------------------------------------------

\begin{aside} % In the aside, each new line forces a line break
\section{contact}
\href{mailto:schillaci@berkeley.edu}{schillaci@berkeley.edu}
~
+1 (206) 930-9189
%~
%\url{linkedin.com/in/coryschillaci}
~
1968 Marin Avenue
Berkeley, CA 94707
\section{programming}
Python, C/C++, Scala, Mathematica, Julia, Matlab
~
scikit-learn, Spark, MLlib, BIDMach, MapReduce, git
\section{selected skills}
data analysis
machine learning
sci. programming
\end{aside}

%----------------------------------------------------------------------------------------
%	EDUCATION SECTION
%----------------------------------------------------------------------------------------

\section{education}

\begin{entrylist}
%------------------------------------------------
\entry
{May 2015}
{PhD, {\normalfont  Physics (Haxton Group)}}
{The University of California, Berkeley}
{Thesis Title: Effective Interactions for Few-Body Physics. 
\begin{itemize}
\item Analytical simplification of three-body interactions in atomic nuclei 
\item Numerically determined spectra of exotic interactions in Bose-Einstein condensates using C++ and Mathematica
\end{itemize}}
%------------------------------------------------
\entry
{June 2009}
{Bachelor of Science, {\normalfont Physics (Reinhardt Group)}}
{The University of Washington}
{Magna cum laude with honors. Minors in mathematics and chemistry. 
\begin{itemize}
\item Built software in C with OpenMP and MPI to simulate dynamics of Bose-Einstein condensates using large-scale parallel computing resources.
\end{itemize}}
%------------------------------------------------
\end{entrylist}

%----------------------------------------------------------------------------------------
%	PUBLICATIONS SECTION
%----------------------------------------------------------------------------------------

\section{publications}

\printbibsection{article} % Print all articles from the bibliography


%----------------------------------------------------------------------------------------
%	RELEVANT COURSEWORK SECTION
%----------------------------------------------------------------------------------------

\section{coursework}

\begin{entrylist}
%------------------------------------------------
\entry
{Spring 2014}
{CS 289A: Introduction to Machine Learning}
{UC Berkeley}
{Theory and practice of classification and regression. Python implementation of models including logistic regression, random forests, neural networks, and boosting methods for classification of MNIST handwritten digits.} 
%------------------------------------------------
\entry
{Spring 2014}
{Info 290T: Data Mining}
{UC Berkeley}
{Practical aspects of data workflow including cleaning, ETL pipelines, distributed computing, modeling, and visualization. Taught by an industry professional.}
%------------------------------------------------
\entry
{Fall 2014}
{CS 281A: Statistical Learning Theory}
{UC Berkeley}
{Theoretical foundations of machine learning including detection and estimation theory, batch and incremental optimization, graphical models, exponential families.}
%------------------------------------------------
\entry
{Jan. 2015}
{Distributed Analytics \& ML with Apache Spark}
{Berkeley Institute for Data Science}
{Apache Spark and MLlib for distributed analysis of large data sets including click-through rate predictions and log analysis.}
%------------------------------------------------
\end{entrylist}

%----------------------------------------------------------------------------------------
%	PROJECTS SECTION
%----------------------------------------------------------------------------------------

\section{projects}

\begin{entrylist}
%------------------------------------------------
\entry
{2014}
{Enhancing Wind Forecasts with Machine Learning}
{}
{Worked with Sail Tactics, LLC to apply machine learning to improve their forecast product.\begin{itemize}
\item Forecast data stored using PostgreSQL and PostGIS on Amazon S3
\item Extensive data acquisition and cleaning from government sources. 
\item Improved RMSE of wind speed predictions by 20\%.
\end{itemize}
}
%------------------------------------------------
\end{entrylist}

%----------------------------------------------------------------------------------------
%	Skills SECTION
%----------------------------------------------------------------------------------------

%\section{skills}
%%Physics grad school allowed me to develop a strong background in mathematics, experience presenting technical concepts to all audiences, and the ability to acquire new knowledge quickly
%\textbf{general:} strong math background, experience critically assessing data and methodologies, rapid acquisition of new tools and concepts, technical presentations \\ 
%\textbf{programming languages:} Python, C/C++, Mathematica, Matlab, Julia\\
%\textbf{programming toolbox:} scikit-learn, Apache Spark, MLlib, git, SQL
%

%----------------------------------------------------------------------------------------

\end{document}